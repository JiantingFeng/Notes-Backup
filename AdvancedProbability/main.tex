\documentclass{amsbook}
 
\usepackage[margin=1in]{geometry} 
\usepackage{amsmath,amsthm,amssymb,amsrefs,amsthm,graphicx,mathtools,tikz,hyperref}

\begin{document}


\title{Graduate Statistics Notes \\ Advanced Probability}
\author{Feng Jianting}
\date{\today}
\maketitle
\tableofcontents
% This is lecture note for \textbf{CUHK-STAT5005}, and the textbook is\cite{durrett2019probability}.
\section{Measure Theory}
\subsection{Probability Spaces}
A \textbf{probability space} is a triple $(\Omega, \mathcal{F}, P)$, where $\Omega$ is a set of "outcomes", $\mathcal{F}$ is a set of "events", and $P: \mathcal{F}\to [0, 1]$ is a function that assigns probabilities to events. $\mathcal{F}$ is called $\sigma$-algebra, which satisties
\begin{enumerate}
	\item if $A\in\mathcal{F}$, $A^C\in\mathcal{F}$, and
	\item if $A_i \in \mathcal{F}$ is a countable sequence, then $\cup_{i}A_i\in \mathcal{F}$.
\end{enumerate}
Without $P$, $(\Omega,\mathcal{F})$ is a \textbf{measurable space}. A \textbf{measure} is a nonnegative countably additive set funciton $\mu: \mathcal{F}\to \mathbb{R}$ with
\begin{enumerate}
	\item $\mu(A)\geq\mu(\emptyset)$, for all $A\in \mathcal{F}$, and
	\item if $A_i$ is a countable sequence of disjoint sets, then $
		      \mu(\cup_i A_i) = \sum_i \mu(A_i)
	      $
\end{enumerate}
If $\mu(\Omega) = 1$, we call $\mu$ is a \textbf{probability measure}, denoted by $P$.
\section{Large Number Theory}

% \newpage
% \bibliographystyle{plain}
% \bibliography{refs.bib}
\end{document}
